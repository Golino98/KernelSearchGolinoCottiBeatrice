\chapter{Improvements}

In this section we will describe the various changes
that we applied to the default kernel search implementation,
with the objective of improving its performance
in solving the MKP instances.

For the sake of completeness, we well also describe the all attemps at improving
the algorithm, thus including the unsuccesful ones.

The effectiveness of the tuning is demonstrated using the selected instances
described in~\ref{subsec:inst} as benchmark.


%\section{Variable Sorting}

%\section{Ejection of Variables from the Kernel}


\section{Repetition Counter}
An observation we could make is that when solving buckets,
after a certain number of iterations the algorithm
repeatedly finds new solutions with the same objective value,
and sometimes it even struggles to find new solutions at all.
In other words, the algorithm gets stuck in \textit{local optima},
from which it's hard to escape.

Our hypothesis is that there are two causes for this:
\begin{enumerate}
    \item The GUROBI solver cannot find a (better) solution for the sub-problem,
    either due to the infeasibility of the problem or because of
    the time limit set for solving each bucket.
    \item The kernel search algorithm, during its improvement phase,
    only accepts a new solution (thus updating the kernel)
    if it improves upon, or is at least equal, to the incumbent one.
    This is done by adding a cutoff constraint, as explained in~\ref{sec:improving-efficiency}.
\end{enumerate}

In order to mitigate this problem we introduced a \textit{repetition counter}
that, during the improvement phase of kernel search,
removes the cutoff constraint for \textit{k} buckets
when the same solution (or no solution) is found for \textit{h} times.

The idea is that this method allows to select if the focus should be on
\textit{diversification} or \textit{intensification},
by appropriately setting parameters \textit{h} and \textit{k}.

A low value for \textit{h} and a high one for \textit{k} allow for
diversification, by adding to the kernel variables that otherwise
may never be selected, and could allow to escape the local optimum.

On the opposite, a high value for \textit{h} and a low one for \textit{k}
switch to focus on intensification, by giving priority to finding
better solutions.

After experimenting with different values for \textit{h} and \textit{k},
we found that keeping them more or less equal allowed for
a reasonable balance between diversification and intensification.
In particular we found that \(h=3\) and \(k=3\) worked quite well
with the test instances we selected.

%\section{Iterative Kernel Search}
