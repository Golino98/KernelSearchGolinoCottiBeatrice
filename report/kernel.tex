\chapter{Kernel Search}\label{ch:kernel-search}
Kernel Search is a heuristic framework used to solve MIP problems, originally presented in~\cite{kernel:2007}.\\
The algorithm has been studied in a few applications, such as the multidimensional knapsack problem
in~\cite{kernel:2010} and the portfolio selection problem in~\cite{kernel:2012}.

The algorithm is based on a few observations that can often be made when solving MIP problems:
\begin{itemize}
    \item in the optimal solution there are only a few non-zero variables;
    \item basic variables in the LP-relaxation are good predictors of non-zero variables in the MIP optimum;
    \item reduced costs are good predictors of the likelihood of a non-basic variable to be in the MIP optimum.
\end{itemize}

Kernel search needs an MIP solver such as GUROBI or CPLEX to solve a set of MIP sub-problems.


\section{Algorithm}
The algorithm has two main phases: an \textit{inizialitation phase}
that builds an initial \textit{kernel set} and a number of \textit{buckets},
and a \textit{improvement phase} that iteratively enlarges the kernel set and improves the solution.

\subsection{Inizialization Phase}
The first step of kernel search is to solve the LP relaxation of the problem.\\
Then, the set of variables of the problem is sorted according to some \textit{sorting criterion}.

Once this is done, the kernel set \(\Lambda\) can be built by selecting the \textit{C}
variables in the sorted set according a \textit{kernel construction criterion},
where \textit{C} is the size of the kernel set.

The next step is to solve MIP(\(\Lambda\)),
which is the original problem restricted to only include the variables in \(\Lambda\):
all the variables not in \(\Lambda\) are set to 0 to exclude them from the problem.\\
The resulting solution will be called \(x^{*}\) and the optimum value \(w^{*}\).\\
It is possible that no solution is found for MIP(\(\Lambda)\): in this case \(w^{*}\) is set to \(-\inf\).

The variables not in \(\Lambda\) are partitioned into a certain number of buckets \textit{nb}, according
to a \textit{bucket construction criterion}.

\subsection{Improvement Phase}
The main objectives of this phase are finding a new improving feasible solution and identifying new variables
to enter the kernel set.

Once the buckets are constructed, the algorithms proceeds with the improvement phase:\\
for each bucket \(B_{i} \ (i=1,\dots,nb)\),
the restricted sub-problem MIP(\(\Lambda \cup B_{i}\)) is solved.

Only the feasible sub-problems with an incumbent solution \(\bar{x^{*}} \geq w^{*}\) are of interest.\\
For such sub-problems the set \(\bar{\Lambda}_{i} \coloneqq \{\text{selected variables in } B_{i}\}\) is built.\\
Then the kernel set is updated: \(\Lambda \coloneqq \Lambda \cup \bar{\Lambda}_{i}\).
Since variable are only added to the kernel, and never removed, its size increases monotonically.

After all the \textit{nb} have been iterated, the algorithm ends.


\section{Parameters}
The kernel search has quite a few configuration parameters that can be used to tune the algorithm for a specific
problem:
\begin{itemize}
    \item \textit{Variable sorting criterion}: used to sort the variables in the initialization phase.
    Ideally, if the sorting criterion is selected appropriately,
    all the significant variables for the problem should be located in \(\Lambda\) and the very first buckets.
    \item \textit{Kernel size C}: the number of items initially selected to construct the kernel.
    \item \textit{Kernel construction criterion}: used to build the kernel.
    The criterion defined in the basic kernel search consists in selecting
    the first C variables in the sorted set, but for certain problems more complex criteria can be used.
    \item \textit{Bucket construction criterion}: how the buckets are constructed.
    Also defines the number and size of the buckets.
\end{itemize}


\section{Improving Efficiency of the Basic Kernel Search}\label{sec:improving-efficiency}
To improve the performance of the basic kernel search, two constraints can be added
when solving each (\(\Lambda \cup B_{i}\)):

\begin{equation}
    \label{eq:impr1}
    \text{value of the objective function} \geq w^{*}
\end{equation}
\begin{equation}
    \label{eq:impr2}
    \sum_{j \in B_{i}} x_{j} \geq 1
\end{equation}

The cutoff constraint~\eqref{eq:impr1} allows only solutions that improve on the current
objective value.
Constraints~\eqref{eq:impr2} ensure that at least one item must be selected from the bucket \(B_{i}\):
such an item will then be included in the new kernel set.


\section{Iterative Kernel Search}\label{sec:iter}
Iterative kernel search is a variation of the basic kernel search where buckets are scrolled more than once.
The algorithm is as follows:
\begin{enumerate}
    \item Execute the basic kernel search
    \item Set \(nb \coloneqq q-1\), where \(q \coloneqq \max\{i:\bar{\Lambda}_{i} \neq \emptyset\}\)
    \item Execute the improvement phase of the algorithm
    \item If a new variable enters the kernel set, \(nb\) is reset to its initial value,
    the improvement phase is executed again and the algorithm is repeated from step 2.
\end{enumerate}

A simpler version of this algorithm repeats step 3 a certain number of times,
skipping steps 2 and 4 entirely.
The \textit{number of iterations} is a configuration parameter.