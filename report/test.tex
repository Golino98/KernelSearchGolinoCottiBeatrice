\chapter{Computational Experiments}
In this section we report the results of test executed on the default kernel search applied to the MKP\@.
The main objective of these experiments was to determine the performance of the default kernel search,
both in terms of quality of the solution found and time required for the execution.


\section{Test Instances}
The test instances for the MKP were kindly provided to us by the authors of~\cite{mkp:2019}.\\
There are in total 2100 test instances, organized in
five directories of increasing complexity:
\begin{itemize}
    \item SMALL
    \item FK\_1
    \item FK\_2
    \item FK\_3
    \item FK\_4
\end{itemize}
SMALL contains 180 instances, while FK\_1, FK\_2, FK\_3, FK\_4 contain 480 instances each.
The instances in the first two directories were already solved to the optimum.
Of those in the last three folders only a few were already solved: the unsolved
ones only had an upper and lower bound.


\subsection{Format of the Instances}
The instances are \texttt{.txt} files whose name of each instance contains different information,
which allows to rapidly understand the parameters of the instance represented.

An example of the name of an instance is the following:
\begin{center}
    \textit{random10\_60\_1\_1000\_1\_12}
\end{center}
The first word is used to indicate if the instance's values are generated randomly or if they are variables of a real
life problem.
The first integer represents the knapsacks number, while the second indicates then number of the items.
The last integer is used to differentiate the instances belonging to the same class.
This instance has 10 knapsacks and 60 items, and it's the \(12^{th}\) in its class instance.

An example of the structure of an instance is shown below:
\[
    \begin{split}
        &\begin{rcases*}
             3 \quad
        \end{rcases*}
        \rightarrow \text{integer that indicates the value of \textit{\textbf{m} (number of knapsacks)}.}\\
        &\begin{rcases*}
             7 \quad
        \end{rcases*}
        \rightarrow \text{integer that indicates the value of \textit{\textbf{n} (number of items)}.}\\
        &\begin{rcases*}
             908 \quad\\
             834\\
             675\\
        \end{rcases*}
        \rightarrow \text{integers that indicate the \textbf{capacity} } c_{i} \text{ of the knapsack i with }
        \space i=1,\dots,m.\\
        &\begin{rcases*}
             264    \quad 430 \quad\\
             606 \quad    945\\
             268 \quad    409\\
             619 \quad    591\\
             958 \quad    839\\
             972 \quad    818\\
             723 \quad    71\\
        \end{rcases*}
        \rightarrow
        \begin{rcases*}
            \text{First column shows the \textbf{weight} } w_{i} \quad \\
            \text{Second column shows the \textbf{profit} } p_{i}.
        \end{rcases*}
        \rightarrow
        \space i=1,\dots,n.
    \end{split}
\]

For most of the instances with the exact solution calculated by the authors of the paper, we were able to
find the exact solution using a GUROBI solver with the following configuration:

\begin{itemize}
    \item \textit{Presolve}: 2 (aggressive presolve)
    \item \textit{Time limit}: 1 hour
    \item \textit{Threads}: 12
    \item \textit{MIPGap}: \num{1e-12}
\end{itemize}

The code can be found at \url{https://github.com/Golino98/EsattoMKP}

\subsection{Selected Instances}
To simplify the testing of the changes to the kernel search algorithm,
we only considered 40 instances: 6 solved to the optimum for each directory,
plus 10 that we weren't able to solve to the best.

Tables~\ref{tab:exact_opt} and~\ref{tab:exact_noopt} represent,
respectively, the optimum values of the 30 instances solved to
the optimum, and the upper and lower bound for the 10
that couldn't be solved.

\begin{table}[]
    \centering
    \resizebox{0.6\columnwidth}{!}{
        \begin{tabular}{@{}lll@{}}
            \toprule
            Directory              & Instance                                   & OPT    \\ \midrule
            \multirow{6}{*}{FK\_1} & random10\_60\_1\_1000\_1\_12               & 23064  \\
            & random10\_100\_2\_1000\_1\_10              & 29800  \\
            & random12\_48\_3\_1000\_1\_16               & 11865  \\
            & random15\_45\_1\_1000\_1\_13               & 15160  \\
            & random15\_75\_3\_1000\_1\_16               & 18321  \\
            & random30\_60\_4\_1000\_1\_14               & 11017  \\ \midrule
            \multirow{6}{*}{FK\_2} & random20\_120\_1\_1000\_1\_12              & 47823  \\
            & random20\_200\_2\_1000\_1\_14              & 53618  \\
            & random24\_96\_3\_1000\_1\_18               & 23912  \\
            & random30\_90\_4\_1000\_1\_20               & 26038  \\
            & random30\_150\_2\_1000\_1\_17              & 44518  \\
            & random60\_120\_4\_1000\_1\_19              & 20463  \\ \midrule
            \multirow{6}{*}{FK\_3} & random30\_180\_1\_1000\_1\_20              & 75618  \\
            & random30\_300\_2\_1000\_1\_4               & 85147  \\
            & random36\_144\_3\_1000\_1\_17              & 39245  \\
            & random45\_135\_4\_1000\_1\_16              & 37005  \\
            & random45\_225\_3\_1000\_1\_15              & 56199  \\
            & random90\_180\_4\_1000\_1\_19              & 30047  \\ \midrule
            \multirow{6}{*}{FK\_4} & random50\_300\_1\_1000\_1\_16              & 119973 \\
            & random50\_500\_2\_1000\_1\_14              & 135583 \\
            & random60\_240\_3\_1000\_1\_14              & 61180  \\
            & random75\_225\_4\_1000\_1\_17              & 60458  \\
            & random75\_375\_2\_1000\_1\_5               & 101917 \\
            & random150\_300\_2\_1000\_1\_12             & 55707  \\ \midrule
            \multirow{6}{*}{SMALL} & probT1\_0U\_R50\_T002\_M010\_N0040\_seed06 & 16510  \\
            & probT1\_0U\_R50\_T002\_M020\_N0060\_seed01 & 26593  \\
            & probT1\_0U\_R50\_T002\_M010\_N0040\_seed05 & 15534  \\
            & probT1\_1W\_R50\_T002\_M010\_N0040\_seed09 & 12724  \\
            & probT1\_1W\_R50\_T002\_M010\_N0060\_seed10 & 19652  \\
            & probT1\_1W\_R50\_T002\_M020\_N0020\_seed10 & 3429   \\
            & probT1\_1W\_R50\_T002\_M020\_N0040\_seed10 & 12405  \\ \bottomrule
        \end{tabular}
    }
    \caption{Exact solutions for the 30 instances already solved to the optimum}
    \label{tab:exact_opt}
\end{table}

\begin{table}[!htbp]
    \centering
    \begin{tabular}{@{}llll@{}}
        \toprule
        Directory              & Instance                      & LB    & UB    \\ \midrule
        \multirow{5}{*}{FK\_3} & random30\_180\_2\_1000\_1\_2  & 47911 & 47966 \\
        & random36\_144\_2\_1000\_1\_20 & 42988 & 43004 \\
        & random45\_135\_3\_1000\_1\_12 & 37680 & 37681 \\
        & random45\_135\_4\_1000\_1\_14 & 52666 & 33756 \\
        & random45\_225\_2\_1000\_1\_20 & 68070 & 68125 \\ \midrule
        \multirow{5}{*}{FK\_4} & random50\_300\_2\_1000\_1\_5  & 80459 & 80483 \\
        & random60\_240\_1\_1000\_1\_17 & 95602 & 95792 \\
        & random60\_240\_2\_1000\_1\_15 & 66497 & 66529 \\
        & random60\_240\_3\_1000\_1\_18 & 59637 & 59648 \\
        & random75\_225\_4\_1000\_1\_20 & 58338 & 58340 \\ \bottomrule
    \end{tabular}
    \caption{Upper and lower bounds for the 10 instances not solved to the optimum}
    \label{tab:exact_noopt}
\end{table}



\section{Performance of the exact algorithm}


\section{Performance of the default kernel search}


\section{Performance of random variable sorting}
