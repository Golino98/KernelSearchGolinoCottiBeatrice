\chapter{Implementation}\label{ch:implementation}
The Java source code for the kernel search was provided during the course,
and it implements a simple iterative kernel search (as explained in~\ref{sec:iter}),
using GUROBI as the MIP solver.
It also includes the improvements described in~\ref{sec:improving-efficiency}.

The code, which can be found alongside its documentation at
\url{https://github.com/Golino98/KernelSearchGolinoCottiBeatrice},
was initially modified to implement the MKP model described in~\ref{sec:model},
and then refactored to improve the code quality and efficiency.\\
The algorithm itself and the configuration parameters were not changed at this time.

The project uses GUROBI 9.5, which is the latest version of GUROBI available at the time of writing.\\
The GUROBI configuration is the following:
\begin{itemize}
    \item \textit{Presolve}: 2 (aggressive presolve).
    \item \textit{MIPGap}: \num{1e-12}.
    \item \textit{Threads}: 12.
\end{itemize}

The kernel search is configured with the following parameters:
\begin{itemize}
    \item \textit{Variable sorting criterion}: sort variables by non-increasing value and non-decreasing reduced cost.
    \item \textit{Kernel size C}: 15\% of the number of variables (rounded to the nearest integer).
    \item \textit{Bucket construction criterion}: iterates through the sorted variables, grouping them in
    buckets of size equal to the 2.5\% of the number of variables (rounded to the nearest integer).
    It's possible for the last bucket to contain fewer items than the previous ones, because the number
    of remaining variables could be inferior to the bucket size.
    \item \textit{Number of iterations}: 2.
\end{itemize}

