\chapter{The Multiple Knapsack Problem}\label{ch:mkp}
The Multiple Knapsack Problem (MKP) is a strongly NP-hard problem
that was described in~\cite{mkp:2019} as follows:
given a set of \(m\) knapsacks with a known \(\text{capacity } c_{i}\)
\((i=1,\dots,m)\) and a set of \(n\) items with
known \(\text{profit } p_{j}\) and \(\text{weight } w_{j}\) \((j=1,\dots,n)\)
the MKP consists in selecting \(m\) disjoint subsets of items
(one subset per knapsack) such that the total weight of the items in the knapsack
does not exceed its capacity, and the profit of the selected items is maximized.


\section{Mathematical Model}\label{sec:model}
There are quite a few formulations for the MKP, with different degrees of
complexity and performance.\\
The model that was used in this project is the classical and most intuitive
one, that uses the following binary variables:
\[x_{ij}=
\begin{cases*}
    1 \quad \text{if item } j \text{ is packed into knapsack } i;\\
    0 \quad \text{otherwise}.
\end{cases*}
\]

The mathematical model is the following:
\begin{equation}
    \label{eq:obj}
    \max{z} = \sum_{i=1}^{m} \sum_{j=1}^{n} p_{j} x_{ij}
\end{equation}
\begin{equation}
    \label{eq:cap}
    \sum_{j=1}^{n} w_{j} x_{ij} \leq c_{i} \qquad i=1,\dots,m
\end{equation}
\begin{equation}
    \label{eq:sel}
    \sum_{i=1}^{m} x_{ij} \leq 1 \qquad j=1,\dots,n
\end{equation}
\begin{equation}
    x_{ij} \in \{0,1\} \qquad i=1,\dots,m \quad j=1,\dots,n
\end{equation}

The objective function~\eqref{eq:obj} maximizes the profit
of the items in the knapsacks.\\
Constraints~\eqref{eq:cap} impose that the capacity of each knapsack is respected,
while constraints~\eqref{eq:sel} ensure that each item
is packed in no more than one knapsack.

We can assume that each knapsack can contain at least one item
(\(\min_{j}\{w_{j}\} \leq \min_{i}\{c_{i}\}\))
and that each item can fit in at least one knapsack
(\(\max_{j}\{w_{j}\} \leq \max_{i}\{c_{i}\}\)).


\section{Real Life Examples}
The MKP has various real life examples and can be used in different fields.

The most straightforward example is the \textbf{cargo organization} for a transport company.
In this scenario, the knapsacks are the trucks or containers.
Their capacity is the effective capacity of the truck/container.\\
The items are the objects that must be delivered:
their weight is the actual weight (or volume) of the goods that need to be shipped,
while the profit is the priority of the delivery.

Another example is the \textbf{project selection}: the knapsacks are
the years available for the development of projects,
and their capacity is the available budget for that year.\\
The items are the available projects: the weight is the cost of the project,
the profit is the monetary gain from the project completion.

More complex examples can be found in the literature.
Applications in the design of computer processors, layout of electronic
circuits and sugar cane alcohol production can be found in~\cite{example:1996}.
Uses in vehicle and container loading are mentioned in~\cite{example:1971}.