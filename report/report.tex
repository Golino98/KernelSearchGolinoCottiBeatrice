\documentclass[a4paper,12pt,oneside]{book} % Formato documento: foglio a4, font size 12, solo fronte
\usepackage[utf8]{inputenc} % Codifica caratteri
\usepackage[lmargin=2cm,rmargin=2cm,tmargin=2.5cm,bmargin=2cm]{geometry} % Bordi pagina come da formato tesi Unibs
\linespread{1.5} % Interlinea
\usepackage{graphicx} % Per poter inserire immagini
\usepackage{float} % Per controllare il posizionamento di immagini e tabelle
\usepackage{csquotes} % Per inserire citazioni
\usepackage{subfig}
\usepackage{fancyhdr} % Per modificare l'intestazione delle pagine
\setlength{\parindent}{1cm} % Rientro paragrafi (???)
\usepackage[backend=biber, sorting=none]{biblatex} % Per poter scrivere la bibliografia su un file a parte
\addbibresource{bibliography.bib} % File contentente la bibliografia

\pagestyle{fancy} % Formato intestazione pagine (???)
\renewcommand{\chaptermark}[1]{\markboth{\thechapter.\ \uppercase{#1}}{}}
\fancyhf{}
\fancyhead[C]{\textbf{\leftmark}}
\fancyfoot[C]{\thepage}
\renewcommand{\headrulewidth}{1pt}
\renewcommand{\footrulewidth}{1pt}
\usepackage[Conny]{fncychap}

\title{Implementation of The Multiple Knapsack Problem in the Kernel Search}
\author{Matteo Beatrice, Luca Cotti, Giacomo Golino}
\date{December 2020}

\begin{document}
    %Frontespizio
    \begin{titlepage}
        \begin{figure}[H] % Logo Unibs
            \centering
            \includegraphics[width=72.4mm]{logo_unibs}
        \end{figure}

        \begin{center}
            \LARGE{\uppercase{Università degli Studi di Brescia}}\\ % Doppio backslash = vai a capo
            \vspace{5mm} % Spazio verticale
            \large{\uppercase{Dipartimento di Ingegneria dell'informazione}}\\
            \vspace{5mm}
            \large{Corso di Laurea Magistrale in Ingegneria Informatica}\\
        \end{center}

        \vspace{15mm}

        \begin{center}
            \LARGE{\textbf{Implementation of The Multiple Knapsack Problem in the Kernel Search}}\\
        \end{center}

        \vspace{10mm}

        \begin{flushleft}
            \large
            \textbf{Docente:}\\
            Ch.ma Prof.sa Renata Mansini
        \end{flushleft}

        \begin{flushright}
            \large
            \textbf{Studenti:}\\
            Matteo Beatrice (000000)\\
            Luca Cotti (719204)\\
            Giacomo Golino (719210)
        \end{flushright}

        \vspace*{\fill} % Spazio bianco per mettere il testo in fondo alla pagina

        \rule{0.8\textwidth}{0.6pt}\\ % Linea nera
        \centering{\Large{Anno Accademico 2021/2022}}
    \end{titlepage}

    %Corpo
    \tableofcontents
    \hyphenpenalty=10000 % Non spezzare le parole a capo

    \chapter{Introduction}

    \chapter{Conclusion}
    \printbibliography
\end{document}