\documentclass[a4paper,12pt,oneside]{book} % Formato documento: foglio a4, font size 12, solo fronte
\usepackage[utf8]{inputenc} % Codifica caratteri
\usepackage[lmargin=2cm,rmargin=2cm,tmargin=2.5cm,bmargin=2cm]{geometry} % Bordi pagina come da formato tesi Unibs
\linespread{1.5} % Interlinea
\setlength{\parindent}{0pt} % No spazio prima dei paragrafi
\setlength{\parskip}{1em}
\setlength{\headheight}{15pt}
\addtolength{\topmargin}{-2.5pt}

\usepackage{graphicx} % Per poter inserire immagini
\usepackage{float} % Per controllare il posizionamento di immagini e tabelle
\usepackage{amsmath} % Per inserire equazioni
\usepackage[hidelinks]{hyperref} % Per inserire URL
\usepackage{xurl} % Per consentire agli URL di spezzarsi a capo
\usepackage{enumitem} % Per poter inserire liste
\usepackage{siunitx} % Necessario per usare \setitemize
\setitemize{noitemsep,topsep=0pt,parsep=0pt,partopsep=0pt} % Per rendere le liste più compatte
\usepackage{booktabs}
\usepackage{multirow}

\usepackage[backend=biber,sorting=none]{biblatex} % Per poter scrivere la bibliografia su un file a parte
\addbibresource{refs.bib} % File contentente la bibliografia

\usepackage{titlesec} % Per modificare il formato dei capitoli
\titleformat{\chapter}[hang]{\normalfont\Huge\bfseries}{\thechapter.}{8pt}{\Huge\bfseries}
\titlespacing*{\chapter}{0pt}{-40pt}{20pt}

\usepackage{fancyhdr} % Per modificare l'intestazione delle pagine
\pagestyle{fancy} % Customizzazione testata e piè di pagina
\renewcommand{\chaptermark}[1]{\markboth{#1}{#1}}
\fancyhf{}
\fancyhead[L]{\thechapter. \leftmark}
\fancyfoot[C]{\thepage}


\title{Implementation of The Multiple Knapsack Problem in the Kernel Search}
\author{Matteo Beatrice, Luca Cotti, Giacomo Golino}
\date{December 2020}

\begin{document}
    \begin{titlepage}
    \begin{figure}[H] % Logo Unibs
        \centering
        \includegraphics[width=72.4mm]{logo_unibs}
    \end{figure}

    \begin{center}
        \LARGE{\uppercase{Università degli Studi di Brescia}}\\ % Doppio backslash = vai a capo
        \vspace{5mm} % Spazio verticale
        \large{\uppercase{Dipartimento di Ingegneria dell'informazione}}\\
        \vspace{5mm}
        \large{Corso di Laurea Magistrale in Ingegneria Informatica}\\
    \end{center}

    \vspace{15mm}

    \begin{center}
        \LARGE{\textbf{Implementation of The Multiple Knapsack Problem in the Kernel Search}}\\
    \end{center}

    \vspace{10mm}

    \begin{flushleft}
        \large
        \textbf{Docente:}\\
        Ch.ma Prof.sa Renata Mansini
    \end{flushleft}

    \begin{flushright}
        \large
        \textbf{Studenti:}\\
        Matteo Beatrice (000000)\\
        Luca Cotti (719204)\\
        Giacomo Golino (719210)
    \end{flushright}

    \vspace*{\fill} % Spazio bianco per mettere il testo in fondo alla pagina

    \rule{0.8\textwidth}{0.6pt}\\ % Linea nera
    \centering{\Large{Anno Accademico 2021/2022}}
\end{titlepage} %Frontespizio

    \tableofcontents % Indice
    \hyphenpenalty=3000 % Rende meno probabile lo spezzamento delle parole a capo

    % I capitoli veri e propri sono su pagine a parte
    \chapter{Introduction}
In the first chapter the Multiple Knapsack Problem is introduced,
along with a few real life examples to better explain the problem.\\
Chapter~\ref{ch:kernel-search} contains a brief description of the Kernel Search algorithm,
along with a few basic improvements to the default algorithm.\\
In chapter~\ref{ch:implementation} we describe the technical
specifications of our implementation of the Kernel Search algorithm,
while in chapter~\ref{ch:computational-experiments} we study
the performance of the implementation and the test instances used as benchmark.\\
Finally in chapter~\ref{ch:improvements} we report all
the methods that we have theorized to better adapt the Kernel Search
algorithm to the Multiple Knapsack Problem.



    \chapter{The Multiple Knapsack Problem}\label{ch:mkp}
The Multiple Knapsack Problem (MKP) is a strongly NP-hard problem
that was described in~\cite{mkp:2019} as follows:
given a set of \(m\) knapsacks with a known \(\text{capacity } c_{i}\)
\((i=1,\dots,m)\) and a set of \(n\) items with
known \(\text{profit } p_{j}\) and \(\text{weight } w_{j}\) \((j=1,\dots,n)\)
the MKP consists in selecting \(m\) disjoint subsets of items
(one subset per knapsack) such that the total weight of the items in the knapsack
does not exceed its capacity, and the profit of the selected items is maximized.


\section{Mathematical Model}\label{sec:model}
There are quite a few formulations for the MKP, with different degrees of
complexity and performance.\\
The model that was used in this project is the classical and most intuitive
one, that uses the following binary variables:
\[x_{ij}=
\begin{cases*}
    1 \quad \text{if item } j \text{ is packed into knapsack } i;\\
    0 \quad \text{otherwise}.
\end{cases*}
\]

The mathematical formulation is the following:
\begin{equation}
    \label{eq:obj}
    \max{z} = \sum_{i=1}^{m} \sum_{j=1}^{n} p_{j} x_{ij}
\end{equation}
\begin{equation}
    \label{eq:cap}
    \sum_{j=1}^{n} w_{j} x_{ij} \leq c_{i} \qquad i=1,\dots,m
\end{equation}
\begin{equation}
    \label{eq:sel}
    \sum_{i=1}^{m} x_{ij} \leq 1 \qquad j=1,\dots,n
\end{equation}
\begin{equation}
    x_{ij} \in \{0,1\} \qquad i=1,\dots,m \quad j=1,\dots,n
\end{equation}
The objective function~\eqref{eq:obj} maximizes the profit
of the items in the knapsacks.\\
Constraints~\eqref{eq:cap} impose that the capacity of each knapsack is respected,
while constraints~\eqref{eq:sel} ensure that each item
is packed in no more than one knapsack.

We can assume that each knapsack can contain at least one item
(\(\min_{j}\{w_{j}\} \leq \min_{i}\{c_{i}\}\))
and that each item can fit in at least one knapsack
(\(\max_{j}\{w_{j}\} \leq \max_{i}\{c_{i}\}\)).


\section{Real Life Examples}
Various real life examples of the use of MKP can be made.

One of the most straightforward ones is the \textit{cargo organization} for a transport company.
In this scenario, the knapsacks are the trucks or containers.
Their capacity is the effective capacity of the truck/container.\\
The items are the objects that must be delivered:
their weight is the actual weight (or volume) of the goods that need to be shipped,
while the profit is the priority of the delivery.

Another example is the \textit{project selection}: the knapsacks are
the years available for the development of projects,
and their capacity is the available budget for that year.\\
The items are the available projects: the weight is the cost of the project,
the profit is the monetary gain from the project completion.

More complex examples can be found in the literature.
Applications in the design of computer processors, layout of electronic
circuits and sugar cane alcohol production can be found in~\cite{example:1996}.
Uses in vehicle and container loading are mentioned in~\cite{example:1971}.
    \chapter{Implementation and Test Instances}


\section{Kernel Search}
The code provided during the course implements a generic kernel search
which uses GUROBI as solver, and has the following default paramenters:
\begin{itemize}
    \item Presolve = 2 (aggressive presolve)
    \item Time limit = 30 seconds
    \item Threads = 12
    \item MIPGap = \num{1e-12}
    \item Sort variables by non-increasing value and non-decreasing reduced cost
    \item Kernel size = 15\% of the number of variables (rounded to the nearest integer)
    \item Bucket size = 2.5\% of the number of variables (rounded to the nearest integer)
    \item Number of iterations = 2
    \item Time limit for each bucket = 5 seconds
\end{itemize}
The code was initially modified to implement the MKP model, and it was then
refactored to improve its quality and efficiency.
The algorithm itself and the parameters were not changed at this time.\\
The code along with extensive documentation can be found at
\url{https://github.com/Golino98/KernelSearchGolinoCottiBeatrice}.


\section{Test Instances}
The test instances for the MKP were kindly provided to us by the authors of~\cite{mkp:2019}.\\
There are in total 2100 test instances, organized in
five directories of increasing complexity:
\begin{itemize}
    \item SMALL
    \item FK\_1
    \item FK\_2
    \item FK\_3
    \item FK\_4
\end{itemize}
SMALL contains 180 instances, while FK\_1, FK\_2, FK\_3, FK\_4 contain 480 instances each.
The instances in the first two directories were already solved to the optimum.
Of those in the last three folders only a few were already solved: the unsolved
ones only had an upper and lower bound.
\newpage


\section{How to read an Instance}
Every instance was generated randomly by the authors of~\cite{mkp:2019}.
The name of each instance contains different information, in order to fastly understand the dimension of the problem represented.
The first word is used to indicate if the instance's values are generated randomly or if they are variables of a real life problem.
The first integer represents the knapsacks number, while the second indicates then number of the items.
The last integer is used to differentiate the instances belonging to the same class.
Here an example:
\textit{\begin{center}
		\text{random10\_60\_1\_1000\_1\_12}
\end{center}}
This instance is generated randomly. It has 10 knapsacks and 60 items. It's the $ 12^{th} $ in the class instance case.

Our test instances are text files, in which are contained the data related to the knapsacks and their \textbf{capacity} and items with their \textbf{profit} and \textbf{weight}. An example of an instance's structure is shown below:

\begin{flushleft}
    $\begin{rcases*}
         3 \quad
    \end{rcases*} \rightarrow $ integer that indicates the value of $ \textit{\textbf{m} (number of knapsacks).}$
\end{flushleft}
\begin{flushleft}
    $\begin{rcases*}
         7 \quad
    \end{rcases*} \rightarrow $ integer that indicates the value of $ \textit{\textbf{n} (number of items).}$
\end{flushleft}

\begin{flushleft}
    $\begin{rcases*}
         908 \quad
         \\834
         \\675
    \end{rcases*} \rightarrow $ integers that indicate the \textbf{capacity}  $ c_{i} $  of the knapsack i with \space $ i=1,\:\dots\:,m. $
\end{flushleft}
\begin{flushleft}
    $\begin{rcases*}
         264    \quad 430 \quad
         \\606 \quad    945
         \\268 \quad    409
         \\619 \quad    591
         \\958 \quad    839
         \\972 \quad    818
         \\723 \quad    71
    \end{rcases*} \rightarrow $
    $\begin{rcases*}
    {\begin{tabular}{@{}l@{}}
         First column shows the \textbf{weight} \(w_{i}\) \\
         Second column shows the \textbf{profit} \(p_{i}\).
    \end{tabular}}
    \end{rcases*} \rightarrow \space$ $ i=1,\:\dots\:,n. $
\end{flushleft}

For most of the instances with the exact solution calculated by the authors of the paper, we were able to
find too the exact using an exact GUROBI solver, with the following configuration:

\begin{itemize}
    \item Presolve = 2 (aggressive presolve)
    \item Time limit = 1 hour
    \item Threads = 12
    \item MIPGap = \num{1e-12}
\end{itemize}

The code can be found at \url{https://github.com/Golino98/EsattoMKP}

To simplify the testing of the changes to the kernel search algorithm,
we only considered 40 instances: 6 solved to the optimum for each directory,
plus 10 that we weren't able to solve to the best.

Tables~\ref{tab:exact_opt} and~\ref{tab:exact_noopt} represent,
respectively, the optimum values of the 30 instances solved to
the optimum, and the upper and lower bound for the 10
that couldn't be solved.

\begin{table}[!htbp]
    \centering
    \resizebox{0.6\columnwidth}{!}{
        \begin{tabular}{@{}lll@{}}
            \toprule
            Directory              & Instance                                   & OPT    \\ \midrule
            \multirow{6}{*}{SMALL} & probT1\_0U\_R50\_T002\_M010\_N0040\_seed05 & 15534  \\
            & probT1\_0U\_R50\_T002\_M020\_N0060\_seed01 & 26593  \\
            & probT1\_1W\_R50\_T002\_M010\_N0040\_seed09 & 12724  \\
            & probT1\_1W\_R50\_T002\_M010\_N0060\_seed10 & 19652  \\
            & probT1\_1W\_R50\_T002\_M020\_N0020\_seed10 & 3429   \\
            & probT1\_1W\_R50\_T002\_M020\_N0040\_seed10 & 12405  \\ \midrule
            \multirow{6}{*}{FK\_1} & random10\_60\_1\_1000\_1\_12               & 23064  \\
            & random10\_100\_2\_1000\_1\_10              & 29800  \\
            & random12\_48\_3\_1000\_1\_16               & 11865  \\
            & random15\_45\_1\_1000\_1\_13               & 15160  \\
            & random15\_75\_3\_1000\_1\_16               & 18321  \\
            & random30\_60\_4\_1000\_1\_14               & 11017  \\ \midrule
            \multirow{6}{*}{FK\_2} & random20\_120\_1\_1000\_1\_12              & 47823  \\
            & random20\_200\_2\_1000\_1\_14              & 53618  \\
            & random24\_96\_3\_1000\_1\_18               & 23912  \\
            & random30\_90\_4\_1000\_1\_20               & 26038  \\
            & random30\_150\_2\_1000\_1\_17              & 44518  \\
            & random60\_120\_4\_1000\_1\_19              & 20463  \\ \midrule
            \multirow{6}{*}{FK\_3} & random30\_180\_1\_1000\_1\_20              & 75618  \\
            & random30\_300\_2\_1000\_1\_4               & 85147  \\
            & random36\_144\_3\_1000\_1\_17              & 39245  \\
            & random45\_135\_4\_1000\_1\_16              & 37005  \\
            & random45\_225\_3\_1000\_1\_15              & 56199  \\
            & random90\_180\_4\_1000\_1\_19              & 30047  \\ \midrule
            \multirow{6}{*}{FK\_4} & random50\_300\_1\_1000\_1\_16              & 119973 \\
            & random50\_500\_2\_1000\_1\_14              & 135583 \\
            & random60\_240\_3\_1000\_1\_14              & 61180  \\
            & random75\_225\_4\_1000\_1\_17              & 60458  \\
            & random75\_375\_2\_1000\_1\_5               & 101917 \\
            & random150\_300\_2\_1000\_1\_12             & 55707  \\ \bottomrule
        \end{tabular}
    }
    \caption{Exact solutions for the 30 instances already solved to the optimum}
    \label{tab:exact_opt}
\end{table}

\begin{table}[!htbp]
    \centering
    \begin{tabular}{@{}llll@{}}
        \toprule
        Directory              & Instance                      & LB    & UB    \\ \midrule
        \multirow{5}{*}{FK\_3} & random30\_180\_2\_1000\_1\_2  & 47911 & 47966 \\
        & random36\_144\_2\_1000\_1\_20 & 42988 & 43004 \\
        & random45\_135\_3\_1000\_1\_12 & 37680 & 37681 \\
        & random45\_135\_4\_1000\_1\_14 & 52666 & 33756 \\
        & random45\_225\_2\_1000\_1\_20 & 68070 & 68125 \\ \midrule
        \multirow{5}{*}{FK\_4} & random50\_300\_2\_1000\_1\_5  & 80459 & 80483 \\
        & random60\_240\_1\_1000\_1\_17 & 95602 & 95792 \\
        & random60\_240\_2\_1000\_1\_15 & 66497 & 66529 \\
        & random60\_240\_3\_1000\_1\_18 & 59637 & 59648 \\
        & random75\_225\_4\_1000\_1\_20 & 58338 & 58340 \\ \bottomrule
    \end{tabular}
    \caption{Upper and lower bounds for the 10 instances not solved to the optimum}
    \label{tab:exact_noopt}
\end{table}



\section{Performance of the exact algorithm}


\section{Performance of the default kernel search}


\section{Performance of random variable sorting}

    \printbibliography[
        heading=bibintoc % Inserisci bibliografia nell'indice
    ]
\end{document}